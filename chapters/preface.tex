\documentclass[../main.tex]{subfiles}

\begin{document}
\section*{What is this book}
이 책은 연세대학교 원주캠퍼스 신입생 C언어 멘토링을 위해 기획 및
제작되었다. 낡고 표준에 맞지 않는 교재, 따라가기 벅찬 강의 진도를
보완하기 위해, 이 책은 예제 중심으로, 그리고 표준과 C 프로그래밍 관례에
부합하는 코드 예시를 이용해 설명하도록 노력하였다. 책의 구성은 그
목적을 따르지만, 예제 중심이라 하여 필요한 내용을 빠뜨리거나 의미없는
내용과 삽화로 장을 낭비하는 일은 결코 없도록 할 것이다.

이 책은 C언어의 기초적 내용을 설명한다. 그러므로 이 책이 설명하는
범위에는 최신 C11 표준 및 고급 I/O (io multiplexing 등), 동시성
프로그래밍, 네트워크 프로그래밍 등의 고급 주제는 포함되어있지 않다. 이
책은 연세대학교 원주캠퍼스 컴퓨터프로그래밍(YHL1015) 및
프로그래밍실습(YHL1013)에서 가르치는 내용들을 다룬다. 그 주제들은
다음과 같다.

\begin{tabular}{ l l l }
  $\bullet$ 기초 터미널 I/O & $\bullet$ 변수 & $\bullet$ 제어문과 연산자\\
  $\bullet$ 함수 & $\bullet$ 배열과 문자열 & $\bullet$ 구조체\\
  $\bullet$ 포인터 & $\bullet$ 파일 I/O & $\bullet$ 전처리기\\
\end{tabular}

\section*{Bad Convention}
비록 이 책이 프로그래밍 그 자체를 처음 배우는 사람들을 위해 기획된
것이지만, C언어는 요즈음에 프로그래밍을 처음 배우는 사람들에게 적합하지
않다. 우선 C언어는 시스템 프로그래밍을 위해 만들어진 언어로, 호환성과
예측가능성을 위해 많은 유연성을 희생한 언어이다. 한 가지 예로, 객체지향
프로그래밍, 제네릭(일반화) 프로그래밍이 보급된지 오래되었지만, 호환성의
문제로 C언어는 class와 generic을 도입하지 않고 있다.\footnote{ 비록
  C11 표준에 \_Generic이 도입되긴 했지만, 이는 C11에서 전처리기에 많은
  기능이 포함되면서 확장된 것일 뿐, 언어 자체의 표현력이 향상되었다고
  보기는 어렵다. 전처리기를 이용해 언어를 확장하려는 시도는 이전부터
  있었다. CObjectSystem(\url{https://github.com/CObjectSystem/COS})
  프로젝트를 참고하라.} 또한 시스템에 의존적인 컴파일 시스템 때문에
라이브러리의 사용법을 가르치기 매우 어렵고, 그로 인해 현실의 문제를
해결하는 것은 거의 모든 교육 현장에서 고려 대상이 아니다. 그 때문에
학생들은 C를 배우고 나서도 ``이것으로 무엇을 할 수 있단 말인가''라는
고민에 빠지게 된다.

다음은 김재우 씨(韓마이크로소프트)가 마이크로소프트웨어에 2003년 기고한
칼럼\footnote{마이크로소프트웨어는 과월호를 제공하지 않아 원전을 확인할
  수 없는 상태이다. 하지만 해당 칼럼은 매우 유명하여 인터넷에서 쉽게
  찾아볼 수 있다. 칼럼의 제목은 \textlangle{}바른 가르침과 배움에 대한
  해답, SICP와 HTDP\textrangle{}이다.}에서 따온 문장이다.

\begin{displayquote}
  선택한 언어가 표현 수준이 낮아 어려운 문제를 푸는 데 많은 사전 작업을
  거쳐야 한다면 학생들은 문제를 푸는 사고를 배운다기보다, 언어의 낮은
  표현 수준을 문제를 표현할 수 있는 수준으로 끌어올리는 데 더 심혈을
  기울여야 한다(대부분의 시스템 프로그래밍 언어가 그렇다). 물론 그 또한
  프로그래밍 교육이라고 할 수야 있겠지만 프로그래밍 사고를 형성하고자
  하는 교육목표와는 거리가 멀다. 다시 말해 프로그래밍 입문 과정에서
  다룰 주제는 아니다.
\end{displayquote}

그 말대로, 프로그래밍 교육은 프로그래밍 언어를 가르치는 것이 아니라,
문제를 해결하는 사고방식을 가르치는 것이다. 거기에 표현수준이 낮은
시스템 프로그래밍 언어(특히 C)는 적합하지 않다. 헌데 우리나라
교육현장에서는 C언어를 첫 프로그래밍 언어로 삼는 것을 당연하게
여긴다. 그 이유는 대개 ``C언어가 가장 가르칠 것이 적기 때문''인데,
이것은 프로그래밍을 가르치는 것과 프로그래밍 언어를 가르치는 것을
혼동하고 있기 때문이다. 이러한 생각은 특이하게도 우리나라에서만
지배적으로 만연해있다. 워싱턴대학교(Java), MIT(Python) 등 수많은 해외
대학의 사례는 이러한 교육 철학이 절대적인 것이 아님을 보여준다.

\section*{그럼에도 불구하고}
그럼에도 불구하고 우리나라의 이러한 교육 환경 안에서, 멘토링 교재로서
이 책의 역할은 교육 현장을 보조하는 역할이기 때문에, 다른 선택의 여지는
없다. 피할수 없다면 즐겁기라도 해야한다. C를 배우면서 가능한 한 흥미를
잃지 않고 중요한 내용을 익힐 수 있도록 이끄는 것이 이 책의 목적이고,
동시에 이 책을 기획하게 된 이유이다.

모쪼록 이 책이 여러분께 C언어 공부의 적당한 시작 지점이 되었으면 한다.
\end{document}