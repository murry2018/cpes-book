\documentclass[../main.tex]{subfiles}
\begin{document}
C언어의 기초 내용을 학습하기 전에, 기본적인 도구들을 익힌다. 다음 내용을 배우게 된다.
\begin{itemize}
\item \texttt{printf}를 이용해 출력하기
\item 주석을 이용해 프로그램을 설명하기
\item 변수를 선언하고, 간단한 연산자들을 이용해 계산하기
\end{itemize}

\label{sec:0.1}
\section{Hello World!}

아래 예제를 컴파일하고 실행하면 콘솔 화면에서 ``Hello World!''라는
문구를 볼 수 있을 것이다.\footnote{이 작은 예제는 C언어의 창시자인
  Dennis Ritchie와 그의 동료인 Brian Kernighan이 쓴 교재 ``The C
  Programming Language''(저자의 두문을 따 K\&R이라고도 불린다)에서
  유래한 역사적인 예제이다. C언어와 K\&R이 가진 역사적 의미를 기리며,
  많은 교재들이 이와 비슷한 예제를 첫 예제로 채택하고 있다.}
\begin{verbatim}
/*--- 예제 0.1 HelloWorld.c ---*/
#include <stdio.h>

int main() {
  printf("Hello World!");
}
\end{verbatim}

\subsection{Details}
\begin{enumerate}
\item \texttt{printf(문자열)}는
  \textbf{문자열}\footnote{문자열(string)이라는 어휘가 생소할 수
    있다. 문자열은 단일 문자들이 늘어선 것을 말한다. C 프로그래밍을
    하면서 어떤 문자든지 모이면 문자열이 될 수 있음을 보게 될
    것이다. 심지어는 \texttt{"} \hspace{5mm} \texttt{"}와 같은 일련의
    공백도 문자열로 취급한다.}을 \textbf{출력}한다.\footnote{출력이라는
    어휘는 print라는 영단어를 직역한 것이다. 이 책 전반에 걸쳐서
    출력이란 말은 대부분 화면에 나타내는 것을 이야기하지만, 프로그래밍
    할 때에는 종종 output의 의미로 쓰이기도 하기 때문에 섣불리 번역을
    고칠 수 없는 어휘이다.} 문자열은 \texttt{"String"}와 같이 따옴표로
  둘러싸인 형태이다.
\item \texttt{/* ... */}는 프로그램을 설명하기 위해 쓰는 것이다. 이러한
  설명을 \textbf{주석}이라고 하며 비슷한 형태의 주석을 앞으로 예제
  곳곳에서 보게 될 것이다. 주석을 쓰는데는 특별한 룰이 없지만, 반드시
  \texttt{/*}와 \texttt{*/} 사이에 써야 한다.
\end{enumerate}

\textbf{주의} : ``\texttt{printf("Hello World!")}''뒤에
세미콜론("\texttt{;}")이 따라 붙음에 주의하라. 앞으로 실습을 하면서
세미콜론을 아주 자주 보게 될 것이다.

이것으로 이 작은 프로그램에 대해 어느정도 설명이 된 것 같지만, 설명하지
않고 남은 것이 있다.

\begin{verbatim}
#include <stdio.h>

int main() {

}
\end{verbatim}

여러분은 모두 이것에 대해 궁금해하겠지만, 아직은 설명할 때가 되지
않았다. 지금은 이것을 프로그램의 기본 골격이라고 생각하자. 즉, 프로그램의
내용은 ``\texttt{int main() \{}''와 ``\texttt{\}}'' 사이에 있어야
한다는 사실만 알아두자. 이것에 대해서는 머지않아 설명할 기회가 있을
것이다.

\section{간단한 수학}
다음은 혈중 알코올 농도를 추정하는 공식이다. \footnote{이 공식은
  스웨덴의 생리학자 Widmark가 20세기 초 제안한 것을 도로교통공단에서
  우리나라의 실정에 맞게 수정한 것이다. Widmark 공식이라고도 하는데,
  음주 운전 사고 발생 시 운전자 음주 상태를 역추산하는데 쓰인다. 법정
  참고자료로 이용되긴 하지만, 사람마다 계산 조건이 조금씩 다르기 때문에
  결정적 증거능력을 갖고 있지는 않다.}

$$ C_{t} = 0.0553 \times \frac{A_{drink} \times A_{concentration}}{ P \times R} - 0.015(t - 1.5) $$

이 때, 각 문자들은 이런 의미를 지니고 있다.

\begin{tabular}{l l}
  $C_{t}$ & :혈중 알코올 농도 \\
  $A_{drink}$ & :음주량(ml) \\
  $A_{concentration}$ & :술의 농도(100\% = 1로 계산)\\
  $P$ & :음주자의 체중(kg) \\
  $R$ & :개인 계수
        (나이와 성별에 따라 다르지만, 주로 남성 0.68, 여성 0.55 이용) \\
  $t$ & :음주 후 경과 시간
        (단, 90분보다 작으면 올바른 추정이 불가능하다. 즉, $t \geq 1.5$)
\end{tabular}

사칙 연산만으로 이루어져 있어 그리 복잡하지 않지만, 손으로 계산하기에는
여간 번거롭지 않다. 마침 컴퓨터는 계산을 위해 만들어진 것이니, 우리도
컴퓨터에게 계산을 시켜보자.

먼저 이 공식을 C언어가 지원하는 형태로 옮겨보면 다음과 같은 코드가
된다(코드가 \texttt{int main() \{ ... \}} 안에 있지 않으므로 온전한
프로그램은 아니다).

\begin{verbatim}
double A_drink, A_concentration, P, R, t, C_t;
A_drink = 500;           /* 음주량 */
A_concentration = 0.05;  /* 술의 농도 */
P = 70;                  /* 음주자의 체중 */
R = 0.68;                /* 남성을 의미. 여성의 경우 0.55 */
t = 2;                   /* 경과 시간 */
/* 혈중 알코올 농도 계산 */
C_t = 0.0553 * (A_drink * A_concentration) / (P * R) - 0.015 * (t - 1.5);
\end{verbatim}

물론 \texttt{A\_drink, A\_concentration, P, R, t}에 넣은 값들은 임의로
집어넣은 것이다. 이 경우, 5도짜리 맥주 500ml를 마신 뒤 2시간이 지난
70kg 남성을 의미한다. 이 값들은 본인이 원하는 대로 바꾸어도
좋다. 참고로, 소주 한병의 용량은 360ml이고, 도수는 16 $\sim$ 19도(즉,
알콜 농도 0.16 $\sim$ 0.19) 정도이다.

\subsection{Details}
\begin{enumerate}
\item 맨 첫 줄에서 \texttt{A\_drink, A\_concentration, P, R, t, C\_t}의
  이름을 미리 \textbf{선언}\small{declaration}해주었다. 이러한 이름들이
  필요할 것이니 기억해 두라는 뜻이다. 이렇게 선언된 이름들을
  \textbf{변수}라고 부르고, 변수에 담긴 값은 \textbf{변수 값}이라고
  부른다.
\item 첫 줄 맨 앞의 \texttt{double}은 후술할 이름들이 \texttt{double}
  형식의 값을 가질 것이라는 것을 의미한다. 뒤에서 자세히 설명할 기회가
  있겠지만, \texttt{double}은 일반적인 숫자를 의미한다.
\item 모든 문장의 끝에 세미콜론(;)이 붙어있음을 주목하라. 세미콜론을
  빠뜨린다면 무시무시한 오류를 맞이하게 될 것이다.
\item 주석은 코드를 설명하기 위한 것일 뿐이니, 아무 곳에 붙어있어도
  상관 없다. 이쁘게 보이게 하려고, 공백을 넣어 주석을 일렬로
  맞추었다. 실제로는 공백을 넣지 않아도 괜찮다.
\item 마지막 줄, 수학 식에서 곱하기 기호($\times$)는
  애스터리스크(\texttt{*})로 바뀌었다.
\end{enumerate}

\label{sec:0.2.2}
\subsection{온전한 C 프로그램으로 고치기}
이것을 완전한 C 프로그램으로 바꾸어 보면 다음과 같다.

\begin{verbatim}
/*--- 예제 0.2 CalcBAC.c ---*/
#include <stdio.h>

int main() {
  double A_drink, A_concentration, P, R, t, C_t;
  A_drink = 500;           /* 음주량 */
  A_concentration = 0.05;  /* 술의 농도 */
  P = 70;                  /* 음주자의 체중 */
  R = 0.68;                /* 남성을 의미. 여성의 경우 0.55 */
  t = 2;                   /* 경과 시간 */
  /* 혈중 알코올 농도 계산 */
  C_t = 0.0553 * (A_drink * A_concentration) / (P * R) - 0.015 * (t - 1.5);
  /* 계산 값 출력 */
  printf("Blood Alcohol Concentration : %f", C_t);
}
\end{verbatim}

컴파일하고 실행하면 다음과 같이 계산된다.
\begin{verbatim}
Blood Alcohol Concentration : 0.021544
\end{verbatim}

변수들 (\texttt{A\_drink, A\_concentration, P, R, t})의 값을
바꾸어가면서, 예상했던 대로 혈중 알코올 농도가 바뀌는 지 살펴보자.

\subsection{Details}
\begin{enumerate}
\item 컴퓨터가 그저 계산을 하는 것 만으로는 충분치 않다. 만약
  \texttt{printf()} 구문을 우리가 쓰지 않았다면, 이 프로그램은 아무
  것도 보여주는 일 없이 종료할 것이다. 
\item \texttt{printf()}가 우리가 섹션 \hyperref[sec:0.1]{0.1}에서
  보았던 것과는 사뭇 다르다. \texttt{printf}는 문자열을 출력하는 것
  말고도 한가지 기능을 더 가지고 있는 데, 바로 문자열을
  \textbf{포매팅}\small{formatting}하는 것이다. 여기서는
  \texttt{"\%f"}를 통해 문자열을 포매팅했다. \texttt{"\%f"}는 문자열과
  함께 들어온 다른 값을 출력한다. 여기서 다른 값이란, \texttt{C\_t}를
  말한다. \texttt{printf}에는 \texttt{"\%f"}을 비롯한 많은 \textbf{서식
    지정자}\small{format specifier}가 있다. 이 책에서는 그 모두를
  다루지 않겠지만, 앞으로 몇 가지를 더 소개하게 될 것이다.
\end{enumerate}

\section{의사 결정}
섹션 \hyperref[sec:0.2.2]{0.2.2}에서 만든 CalcBAC 프로그램은 그저
계산기에 불과하다. 이것만으로는 어떤 의미있는 결론을 도출해 내기
어렵다. 프로그램을 좀 더 의미있게 만들어 보자.

도로교통공단에 따르면, 음주 운전 시 받는 운전면허 행정처분은 다음과 같다.

\begin{tabular}{ c | c }
  혈중 알코올 농도 & 행정처분\\
  \hline
  0.05\% 미만 & 훈방\\
  0.05 $\sim$ 0.10\% & 벌점 100점\\
  0.10\% 이상 & 면허 취소(결격 1년)
\end{tabular}

C의 if와 else는 이러한 조건을 나타내기에 적합하다. 다음 코드를 보자.

\begin{verbatim}
if (C_t < 0.05) {
  printf("훈방");
} else if (C_t < 0.1) {
  printf("벌점 100점");
} else {
  printf("면허 취소(결격 1년)");
}
\end{verbatim}

\subsection{Details}
\begin{enumerate}
\item \texttt{if}는 이런 형태를 가진다.
\begin{verbatim}
if (조건) {
  조건이 참이면 실행할 코드
}
\end{verbatim}
  \texttt{if}는 프로그램의 실행 흐름을 조절할 수 있다. 조건에 따라 어느
  코드가 실행될 지 결정할 수 있기 때문이다. 이렇게 프로그램의 실행
  흐름을 조절하는 요소를 \textbf{제어문}이라고 한다. 그리고 우리는
  \texttt{if}와 \texttt{else}가 제어하는 영역을 각각 \textbf{if 문},
  \textbf{else 문}이라고 부른다.
\item \texttt{else}는 앞의 if문에서 조건이 거짓일 때 실행된다. 바꾸어
  말해, 짝을 이루는 if문이 참이라면 else문은 실행되지 않는다. 따라서,
  첫번째 if문의 조건(\texttt{C\_t \textlangle{} 0.05})이 거짓이라면
  \texttt{C\_t}는 0.05보다 크거나 같다는 사실이 이미 드러난 것이므로,
  구태여 두 번째 if문의 조건(\texttt{C\_t \textlangle{} 0.1})에서
  \texttt{C\_t}가 0.05보다 크거나 같다는 사실을 적어줄 필요가 없다. 맨
  아래의 else에서도 마찬가지이다.
\item 앞의 것이 작다는 뜻의 기호를 ``\verb|<|''로 적은 것이 매우
  자연스럽게 보일 것이다. 비슷하게, 앞의 것이 크다는 뜻의 기호는
  ``\verb|>|''이다. 하지만 ``$\leq$''와 ``$\geq$''는 다르게
  표현된다. 이들은 각각 ``\verb|<=|''와 ``\verb|>=|''로
  표현한다. 이들은 \textbf{관계 연산자}로 불리며, 오직 두개의 대상만을
  비교할 수 있다. 따라서, $ a \leq b \leq c $는 \verb|a <= b <= c|로
  표현하는 게 자연스럽지만, 대부분의 경우 C언어는 이것을 올바로
  해석하지 못한다. 이것의 해결 방법은 나중에 살펴보도록 하자.
\end{enumerate}

\subsection{CalcBAC 프로그램 업그레이드}

어떤 부품을 써야할 지 알게 되었으니, 이제 그 부품을 사용해 CalcBAC
프로그램을 업그레이드 해보자.

\begin{verbatim}
/*--- 예제 0.3 CalcBAC.c ---*/
#include <stdio.h>

int main() {
  double A_drink, A_concentration, P, R, t, C_t;
  A_drink = 500;           /* 음주량 */
  A_concentration = 0.05;  /* 술의 농도 */
  P = 70;                  /* 음주자의 체중 */
  R = 0.68;                /* 남성을 의미. 여성의 경우 0.55 */
  t = 2;                   /* 경과 시간 */
  /* 혈중 알코올 농도 계산 */
  C_t = 0.0553 * (A_drink * A_concentration) / (P * R) - 0.015 * (t - 1.5);
  /* 계산 값 출력 (1): 혈중 알코올 농도 수치 */
  printf("Blood Alcohol Concentration : %f\n", C_t);
  /* 계산 값 출력 (2): 행정 처분 결과 */
  if (C_t < 0.05) {
    printf("훈방");
  } else if (C_t < 0.1) {
    printf("벌점 100점");
  } else {
    printf("면허 취소(결격 1년)");
  }
}
\end{verbatim}

섹션 \hyperref[sec:0.2.2]{0.2.2}에서 선보인 구버전의 CalcBAC에서 코드를
조금 덧붙여 수정했다.

실행 결과는 다음과 같다.
\begin{verbatim}
Blood Alcohol Concentration : 0.021544
훈방
\end{verbatim}

변수들(\texttt{A\_drink, A\_concentration, P, R, t})의 값을
바꾸어가면서, 이만큼 마시고 훈방을 받으려면 얼마나 기다려야 할 지
확인해보자. 물론 사람마다 계산 조건이 조금씩 다르므로 전적으로 믿지는
말고, 재미로만 보도록 하자.

\subsection{Details}
\begin{enumerate}
\item 혈중 알코올 농도 수치를 표시하는 첫 번째 \texttt{printf()} 구문이
  구버전과 비교해 조금 달라졌다. 포매팅된 문자열의 맨 끝을 보면,
  ``\texttt{\textbackslash{}n}''이 추가된 걸 알게 될 것이다. 이것의
  역할은 한 줄을 띄워주는 것이다. 이 특별한 문자는 앞으로 요긴하게
  쓰인다. 잘 기억해두자.
\end{enumerate}

\section{Summary}
\subsection{Terms}
\begin{itemize}
\item \textbf{문자열}은 단일 문자들이 늘어선 것을 말한다. C에서
  문자열은 한 쌍의 따옴표 \texttt{" "}로 둘러싸여 있다.
\item \textbf{출력}이란 화면으로 내보내는 것을 말한다. 더 넓은
  의미로는, Output의 의미로 사용되기도 한다. C에서 화면으로의 출력은
  \texttt{printf()}로 가능하다.
\item \textbf{주석}은 프로그램을 설명하기 위해 사용한다. C에서 주석은
  \texttt{/* ... */} 사이에 있기만 하면 된다.
\item C에서는 앞으로 쓰일 이름들을 미리 \textbf{선언}해 주어야 한다.
\item 값을 갖는 이름들을 \textbf{변수}라고 부른다. C에서 변수는 다음과
  같이 선언한다.
\begin{verbatim}
double variable1;
\end{verbatim}
  혹은, 한번에 여러 개의 변수를 선언하려면,
\begin{verbatim}
double variable_x, variable_y, variable_z;
\end{verbatim}
\item 프로그램의 실행 흐름(순서)를 조절하는 요소를 \textbf{제어문}이라
  한다. \textbf{if 문}과 \textbf{else 문}은 제어문이다.
\item 두 값을 비교하는 연산자들을 \texttt{관계 연산자}라고
  부른다. C에는 \verb|<, >, <=, >=|등의 관계 연산자가 있다.
\end{itemize}
\subsection{Tools}
\begin{itemize}
\item 프로그램의 기본 형식은 다음과 같다.
\begin{verbatim}
#include <stdio.h>

int main() {

}
\end{verbatim}
  모든 코드는 \texttt{int main() \{ ... \}} 안에 있어야 한다.
\item 코드의 모든 문장마다 세미콜론(\texttt{;})을 붙여 주어야
  한다. 예를 들어, 다음과 같은 코드는 적합하지 않다.
\begin{verbatim}
t = 2
printf("Number 't' is : %f", t)
\end{verbatim}
  이는 컴퓨터가 C 코드를 읽을 때 개행이나 들여쓰기 같은 것에 영향을
  받지 않기 때문이다. 예를 들어, 다음과 같은 코드는 완전히 올바른
  코드이다.
\begin{verbatim}
t = 2; printf("Number 't' is: %f", t);
\end{verbatim}
  단지 문장 끝마다 세미콜론(\texttt{;})을 붙여 주기만 하면 된다.
\item \texttt{double}은 일반적인 숫자를 의미한다. 다음과 같은 변수
  선언은 \texttt{variable}이 \texttt{double} 형식의 값을 가진다는 것을
  의미한다.
\begin{verbatim}
double variable;
\end{verbatim}
\item 변수에 값을 담을 때, \texttt{=} 연산자를 이용한다. 수학에서의
  동치 연산자($=$)와 비슷해 보이지만, 아주 같은 역할을 하진 않는다는
  것을 머지 않아 알게 될 것이다.
\begin{verbatim}
variable = 2.5;
\end{verbatim}
  이러한 경우, \texttt{variable}에는 2.5라는 값이 담겨있다.
\item \texttt{printf()}는 주어진 \textbf{서식 문자열}\small{format
    string}을 출력한다. 서식 문자열이란 말은 순전히 C언어를 위해서
  만들어진 용어이다. \texttt{printf()}은 그냥 문자열을 출력할 수도
  있는데, 다음과 같이 하면 된다.
\begin{verbatim}
printf("Hello, World!");
\end{verbatim}
  서식 문자열을 출력하려면 다음과 같이 하면 된다.
\begin{verbatim}
printf("The number: %f", C_t);
\end{verbatim}
  \texttt{\%f}는 그대로 출력되지 않고, \texttt{C\_t}로부터 값을 받아
  그 값을 출력한다.

  \textbf{참고}: \texttt{\%f}를 있는 그대로 출력하고 싶다면
  \texttt{\%}를 표현하는 특별한 형태인 \texttt{\%\%}를 통해서 할 수
  있다. 예를 들어, 다음 코드는 \texttt{\%f}를 화면에 출력한다.
\begin{verbatim}
printf("%%f");
\end{verbatim}
\item C의 문자열에서는 개행(한 줄을 띄우는 것)을 표현하기 위해
  \texttt{"\textbackslash{}n"}을 이용한다. 예를 들어, 다음 코드의
  출력은:
\begin{verbatim}
printf("Hello,\nWorld!\n");
\end{verbatim}
  다음과 같다.
\begin{verbatim}
Hello,
World!
\end{verbatim}
\end{itemize}

\section{연습문제}
\paragraph{문제 0.5.0} 모든 예제(예제 0.1, 0.2, 0.3)들을 컴파일하고,
실행해보라. 특별히, 예제 0.2와 0.3은 변수 값들을 바꾸어보며 그 때마다
결과가 다름을 확인해보자.
\paragraph{문제 0.5.1} 다음 코드를 보자. 이 C 프로그램이 컴파일이
되겠는지 예측해보고, 실제 컴파일 해서 결과를 살펴보자.
\begin{verbatim}
int main() { }
\end{verbatim}
\paragraph{문제 0.5.2} 예제 0.1에서, \verb|#include <stdio.h>|를 빼고
컴파일 해보자. 어떤 에러가 발생하는가? 이 문제와 문제 0.5.1의 결과를
토대로, \texttt{printf()}가 어디에서 오는 것인지 생각해보자.
\paragraph{문제 0.5.3} 다음과 같이 화면에 출력하도록 프로그램을
만들어보라.
\begin{verbatim}
C Programming
for Beginners
\end{verbatim}
\textbf{귀띔} : 어떻게 한 줄을 띄우는 지 모르겠다면, 섹션 0.3.3을
참고하라.
\paragraph{문제 0.5.4} \texttt{double}값을 갖는 네 변수 \texttt{x1,
  y1, x2, y2}를 선언하고, 좌표평면 상의 두 점 $P\langle{}$\texttt{x1,
  y1}$\rangle{}$, $Q\langle{}$\texttt{x2, y2}$\rangle{}$ 사이의 맨해튼
거리\small{manhattan distance}를 계산해 출력하는 프로그램을
만들어라(맨해튼 거리가 무엇인지 모르겠다면, 인터넷을 이용해 맨해튼
거리가 무엇인지 찾아보자. 어렵지 않은 개념이니 금방 이해할 수 있을
것이다). 단, 편의를 위해 \texttt{x2} $>$ \texttt{x1} 이고 \texttt{y2}
$>$ \texttt{y1}이라 하자.

다음은 맨해튼 거리의 몇 가지 예시이다.

\begin{itemize}
\item 두 점 $P\langle{}1, 2\rangle{}$, $Q\langle{}3, 4\rangle{}$ 사이의
  맨해튼 거리는 4이다.
\item 두 점 $P\langle{}2.5, 3.1\rangle{}$, $Q\langle{}9, 7\rangle{}$
  사이의 맨해튼 거리는 10.4이다.
\end{itemize}

\textbf{귀띔} : 맨해튼 거리를 담을 변수 \texttt{distance}를 선언하고,
우리가 예제 0.2에서 했던 것과 같이 \texttt{distance}에 계산 결과를 넣어보라.

\paragraph{문제 0.5.5} 어떤 변수 \texttt{x}에 대해 \texttt{x}의
절댓값을 출력하는 프로그램을 만들어 보라. 절댓값은 다음과 같이 계산된다.

\[
  |x| = \left\{
    \begin{tabular}{l l}
      $x$ & if $x \geq 0$\\
      $-x$ & if $x < 0$
    \end{tabular}
  \right.
\]

\paragraph{문제 0.5.6} 문제 0.5.4에서 더 나아가, \texttt{x2} $>$
\texttt{x1} 이고 \texttt{y2} $>$ \texttt{y1}가 아닐때도 맨해튼 거리를
구하도록 해보자.

\textbf{귀띔} : 두 좌표의 대소 관계에 상관 없이 맨해튼 거리를 계산하기
위해서는 절댓값을 이용해야 한다. 문제 0.5.5에서 배운 내용을 토대로
절댓값을 계산해보자.
\end{document}